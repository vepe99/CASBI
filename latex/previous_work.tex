\chapter{Previous work}

\section{The reconstruction of the Assembly history of the Milky Way}
Inferring the assembly history of the Milky Way is a challenging task, even in the era of the Gaia mission and its 6 dimensional 
phase space data, and the complementary chemical information obtained from the wide-field spectroscopic programs such as the GALAH survey
(De Silva et al. 2015), the H3 survey (Conroy et al. 2019b), APOGEE (Majewski et al. 2017), RAVE (Steinmetz et al. 2006),  SEGUE (Yanny et al. 2009), and 
LAMOST (Cui et al. 2012). The dynamical times of the accreted objects are far longer than the age of the host galaxy, allowing to the 
phase space to retain part of the information on the original orbit parameters. On the other hand, the chemical space is dependent on the star formation history 
of the host galaxy, in particular type II SNe produce $\alpha$-elements and iron with a almost constant ratio, while type Ia SNe produce more efficiently iron. 
The crossmatch between Gaia and spectroscopic data allowed for the of the "Gaia-Sausage-Enceladus" (GSE) Belokurov et al. 2018; Helmi et al. 2018. In order to 
characterize the assembly history \cite{cunninghamReadingCARDsImprint2022} propose to use the "CARDs", the chemical abundance ratio distributions of the
stars, obtained from the FIRE-2 zoom=in cosmological simulations of MW-mass galaxies (Wetzel et al. 2016). Although similar to CASBI in how to leverage 
N-body simulations, this method do not recovers posteriors for the parameters of the accreted objects, but rather considers the host halo as a linear 
combinations of templates CARDs, treating each coefficient as the fraction of mass contribution from the accretion event, and tries to maximize a loss 
function to recover those coefficients.  
\cite{deasonUnravellingMassSpectrum2023} unraveling the mass spectrum using the mean metalicity.
where CASBI comes into play.