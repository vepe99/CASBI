\chapter{Abstract}
Galaxies evolve trough merging events and destroy lower-mass systems over their lifetimes. The contribution that those lower-mass system brings to 
the modern picture has been freezes in stellar halos by the long orbital timescales, making the relicts of this objects retain part of 
their initial progenitor orbit. But dynamical information is not enough to disentangle these components, and the complementary chemical information 
helps to characterized these building blocks. In fact, merging events tend to quench the star formation rate of these objects, making the chemical 
abundance plane (iron abundance against $\alpha$ element abundance) a distinct imprint that retains information on the conditions of formation of their stars, like the total mass and 
the age of the system until the merging event. These theoretical background allows us to attempt to decompose the stellar halo into its components, 
unraveling the merging history. In the modern era of large N-body galaxy simulations, we recast this problem into an SBI pipeline to recover the 
properties of this building block, (e.g. total stellar mass, infall time, ...) using the chemical abundance plane as observables. We therefore present CASBI 
(Chemical Abundance Simulation Based Inferece), a python package to recover the posterior probability of properties of building blocks of Milky Way 
like galaxy's halo. Moreover, CASBI incorporate conditional neural network architectures as generator to obtain observables from parameters, smoothly interpolating on region of the parameters space that weren't fully covered during the N-body simulations.


\newcommand{\vect}[1]{\boldsymbol{#1}}
\pgfplotsset{colormap={whitered}{color(0cm)=(white); color(1cm)=(orange!75!red)}}
\usetikzlibrary{positioning, quotes}
\usetikzlibrary{pgfplots.colorbrewer}

\pgfplotsset{colormap/PuBuGn}
\pgfplotsset{colormap/PuRd}


\begin{tikzpicture}
    \begin{axis}[
        view={45}{60},
        xlabel=$x$,
        ylabel=$y$,
        zlabel=$z$,
        domain=0:5,
        y domain=0:5,
        samples=30
    ]
    \addplot3 [surf] {exp(-((x-2)^2 + (y-2)^2)/0.5) + exp(-((x-4)^2 + (y-4)^2)/0.2)};

    % Marginal projections
    % Projecting onto x-y plane (approximation)
    \addplot3[name path=projXY1, domain=0:5, samples=50, samples y=0, smooth, thick, color=blue] (x, 5, {exp(-(x-2)^2/0.5) + exp(-(x-4)^2/0.2)});

    % Projecting onto y-z plane (approximation)
    \addplot3[name path=projYZ, domain=0:5, samples=50, samples y=0, smooth, thick,  color=green] (0, x, {exp(-(x-2)^2/0.5) + exp(-(x-4)^2/0.2)});
    
    \end{axis}
\end{tikzpicture}